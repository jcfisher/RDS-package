\documentclass[article, shortnames]{jss}

\author{Jacob C. Fisher
        \thanks{Both authors contributed equally to this work.  Author names are ordered alphabetically.}
	       \\Duke University \And 
        Ashton M. Verdery\\University of North Carolina at Chapel Hill
								}
\title{\pkg{RDS2}: A \proglang{Stata} package for respondent-driven sampling (RDS)}

%% for pretty printing and a nice hypersummary also set:
\Plainauthor{Jacob C. Fisher, Ashton M. Verdery} %% comma-separated
\Plaintitle{RDS2: A Stata package for respondent-driven sampling (RDS)} %% without formatting
%\Shorttitle{\pkg{RDS2}: A Capitalized Title} %% a short title (if necessary)

%% an abstract and keywords
\Abstract{
		Respondent-driven sampling (RDS) is a network sampling method used to collect data from hard to reach or hidden populations.  This paper presents the \proglang{Stata} package \pkg{RDS2} for the calculation of estimates and confidence intervals from samples generated by RDS.  \pkg{RDS2} improves on a previous RDS package for \proglang{Stata} by using more recent RDS estimates suggested by \citet{vh2008}, and a bootstrapping method more suited for network samples \citep{salganik2006}.  
}
\Keywords{respondent-driven sampling, RDS, network sampling, \proglang{Stata}}
\Plainkeywords{respondent-driven sampling, RDS, network sampling, Stata} %% without formatting
%% at least one keyword must be supplied

%% publication information
%% NOTE: Typically, this can be left commented and will be filled out by the technical editor
%% \Volume{50}
%% \Issue{9}
%% \Month{June}
%% \Year{2012}
%% \Submitdate{2012-06-04}
%% \Acceptdate{2012-06-04}

%% The address of (at least) one author should be given
%% in the following format:
\Address{
  Jacob C. Fisher\\
  Department of Sociology\\
  Duke University\\
  Durham, NC 27708, United States of America\\
  E-mail: \email{jcf26@duke.edu}\\
  URL: \url{http://bit.ly/TZqNY1}
}
%% It is also possible to add a telephone and fax number
%% before the e-mail in the following format:
%% Telephone: +43/512/507-7103
%% Fax: +43/512/507-2851

%% for those who use Sweave please include the following line (with % symbols):
%% need no \usepackage{Sweave.sty}

%% end of declarations %%%%%%%%%%%%%%%%%%%%%%%%%%%%%%%%%%%%%%%%%%%%%%%


\begin{document}

%% include your article here, just as usual
%% Note that you should use the \pkg{}, \proglang{} and \code{} commands.

\section[Introduction]{Introduction}
Respondent-driven sampling (RDS) has become an increasingly popular sampling method in public health and the social sciences. RDS is a chain-referral sampling method used to estimate population sizes and collect samples in situations where it is difficult to construct a sampling frame.  In particular, RDS is commonly used to sample hard to reach populations, like injecting drug users or sex workers, or hidden populations like jazz musicians \citep{heckathorn97}.

Despite the extensive use of RDS in public health applications, few software tools exist to analyze data collected from samples gathered using RDS.  A stand-alone tool, the RDS analysis tool (RDSAT) \citep{rdsat} calculates population estimates and standard errors from RDS samples, but does not do so from within commonly used statistical software packages.  An \proglang{R} package \pkg{RDS} \citep{r-rds} exists, but its functionality is limited to point estimates.  [Krista's RDS Analyst stand alone tool, the \proglang{R} RDS package which is still in beta testing]  Similarly, a \proglang{Stata} package RDS \citep{stata-rds} exists, but it does not include the most commonly used estimator for RDS samples, the \citet{vh2008} estimator, nor does it include new methods for constructing bootstrap samples, presented by \citet{salganik2006}, which take into account the chain-referral nature of the data.

In this paper, we present the \proglang{Stata} package \pkg{RDS2} which allows users to compute RDS estimates and standard errors from within \proglang{Stata}.  \pkg{RDS2} builds on the previous RDS package for \proglang{Stata} by incorporating the \citet{vh2008} estimator, currently the most commonly used RDS estimator, and the \citet{salganik2006} bootstrapping method, which takes into account the chain-referral nature of the data.  This package is distributed through xxxx.  The key function is xxxx which xxxx.  We illustrate the use of this function with previously published data (cite Giovanna) on sex workers from Liuzhou, China.  The data are contained in the file xxxx which is included with the \pkg{RDS2} package.

\section[Methodology]{Methodology}
RDS samples are collected in a manner similar to snowball sampling.  Currently, there are two estimators which are commonly used with RDS samples: the classical RDS estimator, introduced by (cite), often called the RDS I estimator, and a new RDS estimator introduced by \citet{vh2008}, often called the RDS II estimator.  To calculate standard errors for those estimators, \citet{salganik2006} has introduced a bootstrapping methodology appropriate for chain-referral data.  The following section briefly reviews how both the RDS I and the RDS II estimators, as well as the chain-referral bootstrap calculations, are performed.

\subsection{Classical RDS estimator}
\subsection{RDS II estimator}
\subsection{Bootstrap methods for RDS}

%% Note: If there is markup in \(sub)section, then it has to be escape as above.
\bibliography{rds}


\end{document}
